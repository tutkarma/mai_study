\documentclass[12pt]{article}

\usepackage{fullpage}
\usepackage{multicol,multirow}
\usepackage{tabularx}
\usepackage{ulem}
\usepackage[utf8]{inputenc}
\usepackage[russian]{babel}
\usepackage{amsmath}
\usepackage{amssymb}

\usepackage{titlesec}

\titleformat{\section}
  {\normalfont\Large\bfseries}{\thesection.}{0.3em}{}

\titleformat{\subsection}
  {\normalfont\large\bfseries}{\thesubsection.}{0.3em}{}

\titlespacing{\section}{0pt}{*2}{*2}
\titlespacing{\subsection}{0pt}{*1}{*1}
\titlespacing{\subsubsection}{0pt}{*0}{*0}
\usepackage{listings}
\lstloadlanguages{Lisp}
\lstset{extendedchars=false,
    breaklines=true,
    breakatwhitespace=true,
    keepspaces = true,
    tabsize=2
}
\begin{document}


\section*{Отчет по лабораторной работе №\,1
по курсу \guillemotleft  Функциональное программирование\guillemotright}
\begin{flushright}
Студент группы М8О-307 МАИ \textit{Довженко Анастасия}, \textnumero 7 по списку \\
\makebox[7cm]{Контакты: {\tt tutkarma@gmail.com} \hfill} \\
\makebox[7cm]{Работа выполнена: 15.03.2019 \hfill} \\
\ \\
Преподаватель: Иванов Дмитрий Анатольевич, доц. каф. 806 \\
\makebox[7cm]{Отчет сдан: \hfill} \\
\makebox[7cm]{Итоговая оценка: \hfill} \\
\makebox[7cm]{Подпись преподавателя: \hfill} \\

\end{flushright}

\section{Тема работы}
Примитивные функции и особые операторы Common Lisp.

\section{Цель работы}
Научиться вводить S-выражения в Лисп-систему, определять переменные и функции, работать с условными операторами, работать с числами, использую схему линейной и древовидной рекурсии.

\section{Задание (вариант №1.34)}



Поле шахматной доски определяется парой натуральных чисел, каждое из которых не превосходит восьми:
\begin{itemize}
\item первое число - номер вертикали (при счете слева направо).
\item второе - номер горизонтали (при счете снизу вверх).
\end{itemize}

Определите на языке Коммон Лисп функцию-предикат с четырьмя параметрами - натуральными числам {\tt k, l, m, n}, каждое из которых не превосходит восьми.\\

{\tt k, l}\\
Задают поле, на котором расположена фигура - ладья.

{\tt m, n}\\
Задают поле, куда она должен попасть.\\

Функция должна возвращать\\
{\tt T},\\
если ладья {\tt (k,l)} может попасть на поле {\tt (m,n)} за один ход;\\

{\tt i, j}\\
два значения с помощью {\tt values}, если ладья {\tt (k,l)} может попасть на поле {\tt (m,n)} за два хода через поле {\tt (i,j)}.



\section{Оборудование студента}
Ноутбук Asus UX310U, процессор Intel Core i7-6500U CPU 2.50GHz x 4, память: 8Gb, разрядность системы: 64.

\section{Программное обеспечение}
ОС Ubuntu 16.04 LTS, компилятор clisp, текстовый редактор Sublime Text 3.

\section{Идея, метод, алгоритм}
Ладья может попасть из {\tt (k,l)} в {\tt (m,n)} за один ход, если она расположена на той же вертикали или/и горизонтале, что и конечное поле. Отсюда получаем условие  {\tt k == m or l == n}. Во всех остальных случаях ладья попадет в конечное поле через <<транзитную>> клетку, которую можно выбрать двумя способами: либо это будет клетка {\tt (m,l)}, либо клетка  {\tt (k,n)}. В своем решении я выбрала второй вариант. 

\section{Сценарий выполнения работы}

\section{Распечатка программы и её результаты}

\subsection{Исходный код}
\lstinputlisting{./main.lisp}

\subsection{Результаты работы}
\begin{lstlisting}
(castle-moves 1 1 1 1)
"T"
(castle-moves 4 5 7 8)
4 ;
8
(castle-moves 7 7 2 4)
7 ;
4
(castle-moves 2 6 6 6)
"T"

\end{lstlisting}

\section{Дневник отладки}
\begin{tabular}{|c|c|c|c|}
\hline
Дата & Событие & Действие по исправлению & Примечание \\
\hline
\end{tabular}

\section{Замечания автора по существу работы}
Работа показалась мне слишком простой с точки зрения программирования. 

\section{Выводы}
При выполнении работы я вспомнила синтаксис языка. Мне кажется, эта работа была больше направлена на аналитическое решение (хоть и простое), нежели на программное. Основные сложности были связаны со средой разработки, потому что мне не хотелось использовать IDE, онлайн-компиляторы по непонятной мне причине не выводили все значения, возвращаемые функцией, и только из консоли это заработало более-менее приемлимо.

\end{document}