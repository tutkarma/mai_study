\documentclass[12pt]{article}

\usepackage{fullpage}
\usepackage{multicol,multirow}
\usepackage{tabularx}
\usepackage{ulem}
\usepackage[utf8]{inputenc}
\usepackage[russian]{babel}
\usepackage{listings}
\usepackage{hyperref}


\begin{document}

\section*{Лабораторная работа №\,1 по курсу криптографии}

Выполнила студентка группы М8О-307Б \textit{Довженко Анастасия}.

\subsection*{Условие}
Разложить каждое из чисел n1 и n2 на нетривиальные сомножители.\\
Вариант 7.\\
n1=108762353292448487441247663685513658893167646930627178946128889967643172154127
n2=161176556914880485624286738425868071985001028629819120463515415294204321972904\\
47526886147483136114545465725205417369977940016871273001825655775233013745768986374\\
65463079329544247774787283512154983161737116562645744234565727709746364114005583231\\
54796702302541456941312244732804041697084530943221753072243334150616687905813526765\\
27375610862399155982339310065668240742080964683365204046938632685331174477299911625\\
79236036416014409092228354404809885779998800076550137

\subsection*{Метод решения}

Был использован Ро-алгоритм Полларда. Изначально ищем все простые делители исходного числа, затем, как несложно догадаться получаем все существующие делители перемножением простых делителей. Как ищутся простые делители? Случайным образом выбирается число $x$, на каждой итерации вычисляется значение функции $f^{i}(x)$. В качестве функции взята $f(x)= ax^{2} + b$ для случайных $a$ и $b$. На $i$-ом шаге получаем значения $x_{i} = f^{i}(x_{0}) (mod n), y_{i} = x_{2i} = f^{2i}(x_{0}) (mod n)$. $gcd(abs(x - y), n)$ даст нетривиальный сомножитель n. Функция выбирается каждый раз, когда количество итераций превышает $O(\sqrt{n})$. Проверка простоты числа осуществляется с помощью теста Миллера-Рабина, который позволяет выполнять проверку быстрее.\\

Полученная реализация, проработав ночь, все еще искала сомножители первого числа, поэтому было решено воспользоваться готовой реализацией метода решета числового поля -- \href{https://github.com/radii/msieve}{msieve
}
По совету старших коллег при факторизации второго числа был использован хак: первый множитель находится как НОД с числом другого варианта, а второй множитель -- делением. 


\subsection*{Результат работы программы}
\begin{lstlisting}
karma@mydruzhok:~/Downloads/msieve-1.53$ ./msieve 10876235329244848744
1247663685513658893167646930627178946128889967643172154127

sieving in progress (press Ctrl-C to pause)
36960 relations (19767 full + 17193 combined from 190013 partial), need 
36567
sieving complete, commencing postprocessing
karma@mydruzhok:~/Downloads/msieve-1.53$ cat msieve.log 
Sat Mar  9 21:38:59 2019  
Sat Mar  9 21:39:00 2019  
Sat Mar  9 21:39:00 2019  Msieve v. 1.53 (SVN unknown)
Sat Mar  9 21:39:00 2019  random seeds: d7a92415 3ab6e5a2
Sat Mar  9 21:39:00 2019  factoring 1087623532924484874412476636855136
58893167646930627178946128889967643172154127 (78 digits)
Sat Mar  9 21:39:00 2019  no P-1/P+1/ECM available, skipping
Sat Mar  9 21:39:00 2019  commencing quadratic sieve (78-digit input)
Sat Mar  9 21:39:00 2019  using multiplier of 23
Sat Mar  9 21:39:00 2019  using generic 32kb sieve core
Sat Mar  9 21:39:00 2019  sieve interval: 12 blocks of size 32768
Sat Mar  9 21:39:00 2019  processing polynomials in batches of 17
Sat Mar  9 21:39:00 2019  using a sieve bound of 921203 (36471 primes)
Sat Mar  9 21:39:00 2019  using large prime bound of 92120300 (26 bits)
Sat Mar  9 21:39:00 2019  using trial factoring cutoff of 26 bits
Sat Mar  9 21:39:00 2019  polynomial 'A' values have 10 factors
Sat Mar  9 21:40:56 2019  36960 relations (19767 full + 17193 combined 
from 190013 partial), need 36567
Sat Mar  9 21:40:56 2019  begin with 209780 relations
Sat Mar  9 21:40:56 2019  reduce to 51987 relations in 2 passes
Sat Mar  9 21:40:56 2019  attempting to read 51987 relations
Sat Mar  9 21:40:57 2019  recovered 51987 relations
Sat Mar  9 21:40:57 2019  recovered 39223 polynomials
Sat Mar  9 21:40:57 2019  attempting to build 36960 cycles
Sat Mar  9 21:40:57 2019  found 36960 cycles in 1 passes
Sat Mar  9 21:40:57 2019  distribution of cycle lengths:
Sat Mar  9 21:40:57 2019     length 1 : 19767
Sat Mar  9 21:40:57 2019     length 2 : 17193
Sat Mar  9 21:40:57 2019  largest cycle: 2 relations
Sat Mar  9 21:40:57 2019  matrix is 36471 x 36960 (5.4 MB) with weight
1107501 (29.96/col)
Sat Mar  9 21:40:57 2019  sparse part has weight 1107501 (29.96/col)
Sat Mar  9 21:40:57 2019  filtering completed in 3 passes
Sat Mar  9 21:40:57 2019  matrix is 25132 x 25196 (4.0 MB) with weight
838109 (33.26/col)
Sat Mar  9 21:40:57 2019  sparse part has weight 838109 (33.26/col)
Sat Mar  9 21:40:57 2019  saving the first 48 matrix rows for later
Sat Mar  9 21:40:57 2019  matrix includes 64 packed rows
Sat Mar  9 21:40:57 2019  matrix is 25084 x 25196 (2.4 MB) with weight
575627 (22.85/col)
Sat Mar  9 21:40:57 2019  sparse part has weight 373813 (14.84/col)
Sat Mar  9 21:40:57 2019  commencing Lanczos iteration
Sat Mar  9 21:40:57 2019  memory use: 2.4 MB
Sat Mar  9 21:41:02 2019  lanczos halted after 398 iterations (dim =
25074)
Sat Mar  9 21:41:02 2019  recovered 13 nontrivial dependencies
Sat Mar  9 21:41:02 2019  p39 factor: 26095128986248577264472725816265
2873363
Sat Mar  9 21:41:02 2019  p39 factor: 41679178267240329566284173772868
5758229
Sat Mar  9 21:41:02 2019  elapsed time 00:02:02
karma@mydruzhok:~/Downloads/msieve-1.53$ cd ~/mai_study/Crypto/lab1/
karma@mydruzhok:~/mai_study/Crypto/lab1$ python main.py 
Original number: 161176556914880485624286738425868071985001028629819
12046351541529420432197290447526886147483136114545465725205417369977
94001687127300182565577523301374576898637465463079329544247774787283
51215498316173711656264574423456572770974636411400558323154796702302
54145694131224473280404169708453094322175307224333415061668790581352
67652737561086239915598233931006566824074208096468336520404693863268
53311744772999116257923603641601440909222835440480988577999880007655
0137
Factors:
16339769606582107468090265599682557015970679523604590652155946096257
85190788561057256489685565690727114066165297231829395018127947226623
66814883631619640072792920581850719503493330646427755230896373119814
69057198581127811557725160954236258017514857831373908089824469638166
5260084479643389434792645421908712913
9.86406545475122e+153
Time: 0:00:00.000077
\end{lstlisting}

\subsection*{Выводы}
Факторизация целых чисел обеспечивается основной теоремой арифметики. По основной теореме арифметики каждое натуральное число имеет единственное разложение на простые множители. Существует множество алгоритмов факторизации целого, с помощью которых можно факторизовать любое натуральное число до состава его простых множителей с помощью рекуррентных формул. Однако, для очень больших чисел эффективный алгоритм пока неизвестен.\\ 
Факторизация больших чисел является задачей большой сложности. Не существует никакого известного способа, чтобы решить эту задачу быстро. Её сложность лежит в основе некоторых алгоритмов шифрования с открытым ключом, таких как RSA. Множество областей математики и информатики находят применение в решении этой задачи. Среди них: эллиптические кривые, алгебраическая теория чисел и квантовые вычисления. 

\subsection*{Листинг программного кода}
\begin{lstlisting}[language=Python]
from random import randint
from math import gcd, pi
from functools import reduce
from datetime import datetime

CHECK_NUM = 159875654421086081200268325250466663128403853515497934
091096482467392357863922639791813442919273700585418817797705917785
824385599080398127566569091297553409104136170184346557810173386347
978168079165595957832044210837163404837431352420219319869489453645
247164686882514474301445295791274392023995447353437442264774802016
530676937939619004459951311039306246130283924435675474106532077501
151477472315586373159518289282279070984329637507527265190264146050
4103291775361


def f(x, a, b):
    return a * x ** 2 + b


def is_prime(N):
    if N in (0, 1):
          return False
    if N == 2:
        return True
    if N % 2 == 0:
        return False
    s = N - 1
    while s % 2 == 0:
        s //= 2
    for i in range(50):
        a = randint(1, N-1)
        exp = s
        mod = pow(a, exp, N)
        while exp != N - 1 and mod != 1 and mod != N - 1:
            mod = mod * mod % N
            exp *= 2
        if mod != N - 1 and exp % 2 == 0:
            return False
    return True


def find_factor(n):
    maxiterssq = pi / 4 * n
    x = randint(1, n - 1)
    y = x
    d = 1
    iters = 0
    a = randint(1, n - 1)
    b = randint(1, n - 1)
    while d in (1, n):
        if iters ** 2 > maxiterssq:
            a = randint(1, n - 1)
            b = randint(1, n - 1)
            x = randint(1, n - 1)
            y = x
            iters = 0
        x = f(x, a, b) % n
        y = f(f(y, a, b), a, b) % n
        d = gcd(abs(x - y), n)
        iters += 1
    return d


def find_prime_factor(n, factors):
    if is_prime(n):
        factors.append(n)
    else:
        tmp = n // find_factor(n)
        find_prime_factor(tmp, factors)


def factor(n, factors):
    while n % 2 == 0:
        factors.append(2)
        n //= 2
    while n % 3 == 0:
        factors.append(3)
        n //= 3
    while n > 1:
        find_prime_factor(n, factors)
        n //= factors[-1]


def find_all_factors(prime_factors, all_factors):
    if len(prime_factors) == 0:
        all_factors.append(1)
    elif len(all_factors) == 0:
        all_factors.append(1)
        all_factors.append(prime_factors[0])
    for i in range(1, len(prime_factors)):
        tmp = []
        for f in all_factors:
            if f * prime_factors[i] not in all_factors:
                tmp.append(f * prime_factors[i])
        all_factors += tmp
    all_factors.sort()


def get_info(num, factors, start_time):
    print("Original number: {0}".format(num))
    print("Factors:")
    print(*factors, sep='\n')
    print("Time: {0}".format(datetime.now() - start_time))


def pollard_rho(n):
    factors = []
    factor(n, factors)
    factors.sort()
    all_factors = []
    find_all_factors(factors, all_factors)
    return all_factors


if __name__ == '__main__':
    n = 0
    with open('test1', 'r') as file:
        n = int(file.read())
    start_time = datetime.now()
    factors = pollard_rho(n)
    factors = factors[1:len(factors) - 2]
    get_info(n, factors, start_time)

    factors.clear()
    with open('test2', 'r') as file:
        n = int(file.read())
    start_time = datetime.now()
    factors.append(gcd(n, CHECK_NUM))
    factors.append(n / factors[0])
    get_info(n, factors, start_time)

\end{lstlisting}

\end{document}
