\section{Выполнение работы}

После запуска программы видим окно с классическим фракталом множества Мандельброта

\includegraphics[scale=0.5]{img/07_01.png}\\

Черный цвет в середине показывает, что в этих точках функция стремится к нулю - это и есть множество Мандельброта. За пределами этого множества функция стремится к бесконечности. А самое интересное это границы множества. Они то и являются фрактальными. На границах этого множества функция ведет себя непредсказуемо - хаотично.

Точки, принадлежащие множеству Мандельброта, соответствуют связным множествам Жюлиа, а точки не принадлежащие — несвязным.

Увеличив масштаб, я обнаружила часть множества Мандельброта, точки которого соответствуют связному множеству Жюлиа.

\includegraphics[scale=0.5]{img/07_02.png}\\

Когда $с$ выйдет за границу множества Мандельброта, сопутствующее ему множество Жюлиа как бы взорвется, превратившись в Канторову пыль. Эта пыль становится все мельче с удалением точки $с$ от множества Мандельброта.


Поменяем параметры формулы mandel на $Imaginary Perturbation of Z(0) = 0.35$

\includegraphics[scale=0.5]{img/07_03.png}\\

Получившееся изображение:

\includegraphics[scale=0.5]{img/07_04.png}\\

К сожалению, на моем компьютере отсутствуют клавиши PageUp и PageDown. К тому же в программе FractInt нельзя менять шорткаты.

Рассчитаем неподвижную траекторию. Должно выполняться равенство $z_{n + 1} = z_{n}$.

$$ z_{n + 1} = R(z_{n + 1}) + i \cdot I(z_{n + 1}) = z_{n}^{2} + c = (R(z_{n}) + i \cdot I(z_{n}))^{2} + (R(c) + i \cdot I(c))$$


\begin{equation*}
 \begin{cases}
   R(z_{n + 1}) = R^{2}(z_{n}) - I^{2}(z_{n}) + R(c)
   \\
   I(z_{n + 1}) = 2 \cdot R(z_{n}) \cdot I(z_{n}) + I(c)
 \end{cases}
\end{equation*}


Подставим $z_{0} = 0.35i$:

\begin{equation*}
 \begin{cases}
   R(z_{1}) = 0^{2} - 0.35^{2} + R(c) = 0
   \\
   I(z_{1}) = 2 \cdot 0 \cdot 0.35 + I(c) = 0.35
 \end{cases}
\end{equation*}

$$
c = 0.1225 + 0.35i
$$

Рассмотрим точку, для которой последовательность будет ограничена. Эта точка из множества Мандельброта, например $c = 0.01 - 0.1i$. Запустим итерационный процесс, пока условие окончание итерационного процесса не выполнится (полученные значения не будут лежать достаточно близко друг к другу).

\begin{lstlisting}[language=Python]
C = (0.01-0.1j)
Z_0 = 0.35j
Iteration #1
Z_1 = (-0.11249999999999999-0.1j)
Iteration #2
Z_2 = (0.01-0.1j)
Iteration #3
Z_3 = (0.012656249999999996-0.07750000000000001j)
Iteration #4
Z_4 = (9.999999999999766e-05-0.10200000000000001j)
Iteration #5
Z_5 = (0.004153930664062498-0.10196171875j)
Iteration #6
Z_6 = (-0.00040399000000000164-0.10002040000000001j)
Iteration #7
Z_7 = (-0.0003789369504922629-0.10084708382015228j)
\end{lstlisting}

\pagebreak

