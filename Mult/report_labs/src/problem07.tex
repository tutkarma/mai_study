\CWHeader{Лабораторная работа \textnumero 7. Множества Жюлиа и Мандельброта}


\textbf{Цели:} Изучить процесс построения алгебраических фракталов и результаты их визуализации.

\textbf{Задание:}
\begin{enumerate}

\item В среде программы FractInt рассмотреть классическую формулу $z(n + 1) = z(n)^{2}+c$ (mandel). Увеличить масштаб, с помощью правой кнопки мыши изучить вид соответствующих множеств Жюлиа. В отчете привести пример связного множества Жюлиа, Канторовой пыли.

\item В качестве параметров формулы mandel задать $Imaginary Perturbation of Z(0) = 0.05 \cdot 7$

\item Подобрать для формулы удобный вид с помощью клавиш позиционирования PgUp и PgDown, клавиш палитры + и -. Привести изображение в отчете.

\item Рассчитать неподвижную траекторию, привести пример точки, для которой последовательность будет ограничена.
\end{enumerate}

\textbf{ПО:} xfractint

\pagebreak