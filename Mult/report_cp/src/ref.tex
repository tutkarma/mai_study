\section{Реферат}

В данном курсовом проекте было использовано следующее программное обеспечение:

VistaPro – программа для фрактальной генерации 3D ландшафта. С помощью нее можно создавать случайный ландшафт и изменять его. Например можно менять освещения, уровень моря, облаков, растительности, снега, реки, озёра. Vista может загружать и сохранять выходные изображения в форматах PCX, BMP, JPG и Targa. Файлы PCX также могут быть импортированы в виде возвышений и основных цветов, что позволяет создавать ландшафты в других редакторах изображений. Деревья могут быть размещены на ландшафтах как 2D или 3D объекты. В 2D деревья всегда обращены к камере и быстро генерируются. Трехмерные деревья создаются с использованием фракталов, и им можно изменять изгиб ветвей, чтобы они выглядели более сложными.

Adobe Premiere Pro – профессиональная программа для видеомонтажа компании Adobe Systems. В ней представлено большое множество различных инструментов для редактирования видео и даже аудио. То что в данной программе проводится монтаж фильмов, уже говорит о ее статусе среди подобных продуктов. Среди предоставляемых возможностей например имеется: склейка и обрезание видео и аудио дорожек, наложение эффектов на видео, создание титров разных видов и многое другое. Рендер видео можно производить во множество различных форматов с различными кодеками. По заданию нужно было использовать AVI формат с кодеком использующим либо DCT, либо Wavelet.

Сначала была создана анимированная последовательность облета виртуального мира, созданного с помощью VistaPro. После создания ландшафта, гор, снега, солнца, реки с водопадом, моря и деревье, также была изменена цветовая палитра нескольких элементов ландшафта для создания эффекта «чужой планеты». Затем произведен облет данного виртуального мира и выгрузка его в формат AVI без сжатия.

Далее имеющийся результат был импортирован в среду Adobe Premiere Pro. И после выполнения всех пунктов, требуемых при обработке видео: титры, эффекты, замедление на крупном плане, наложение видео поверх с вырезанным фоном, был дважды произведен рендеринг. Сначала без сжатия, а затем с использованием кодека DV NTSC.

В формате DV NTSC используется 8-битный цифровой компонентный видеосигнал с разрешением 720х420 пикселей и частотой выборке субдискретизации 4:1:1. Для уменьшения избыточности сигнала используется внутрикадровая компрессия на основе дискретного косинусного преобразования (ДКП). Коэффициент компрессии сигнала — 5:1. Скорость потока данных: 25 Мбит/с видео, 1,5 Мбит/с аудио и 3,5 Мбит/с служебной информации. Поддерживается запись двух каналов звукового сопровождения с частотой дискретизации аудиосигнала 48 кГц при 16-битном квантовании или четырёх каналов звука с параметрами 32 кГц/12 бит. В служебной области производится запись даты и времени.

Схема сжатия видео данным кодеком в общем виде изображена на следующей схеме:

\includegraphics[scale=5]{img/01.png}\\

Таким образом сначала происходит формирование блоков кадров,  состоящих из четырёх яркостных подблоков и двух цветоразностных блоков. Перемешивание заключается в том, что специальным псевдослучайным образом выбираются 5 блоков, образующих сегмент видео.

После этого применяется дискретное косинусное преобразование к каждому блоку.

Являясь ключевым шагом алгоритма сжатия, дискретное косинусное преобразование представляет собой разновидность преобразования Фурье и, также как и последнее, имеет обратное преобразование (ОДКП). Как и другие преобразования, дискретное косинусное преобразование пытается декоррелировать изображение. После декорреляции каждый коэффициент преобразования может кодироваться независимо без потери эффективности сжатия. Таким образом ДКП позволяет выбрать информацию, которую можно безболезненно отбросить, не внося серьезных искажений в картинку. Если рассматривать изображение как совокупность пространственных волн, где оси X и Y соответствуют ширине и высоте картинки, а по оси Z откладываются значения цвета соответствующих пикселей, то можно перейти от пространственного представления картинки к ее спектральному представлению и обратно. ДКП преобразует матрицу пикселей размера N x N в матрицу частотных коэффициентов соответствующего размера. В получаемой матрице низкочастотные компоненты расположены ближе к левому верхнему углу, а более высокочастотные смещаются вправо вниз. В связи с тем, что основная часть графических образов на экране состоит из низкочастотной информации, используя полученную матрицу можно дифференцированно отбрасывать наименее важную информацию с минимальными визуальными потерями.

Формула одномерного ДКП:

\includegraphics[scale=0.7]{img/02.png}\\

Формула обратного преобразования:

\includegraphics[scale=0.7]{img/03.png}\\

Таким образом, первый коэффициент преобразования представляет собой среднее значение последовательности. В литературе его часто называют DC коэффициентом, а остальные коэффициенты – AC.

Рассмотрим набор базисных функций ДКП. Обратим внимание, что эти базисные функции являются ортогональными. Следовательно, умножение любого элемента на другой элемент с последующим суммированием по всем точкам выборки дает нулевое (скалярное) значение. Ортогональность функций дает независимость, то есть ни одна из базовых функций не может быть представлена в виде комбинации других базовых функций, следовательно избыточность отсутствует. 

Так как мы работает с изображением, то нам нужно ДКП в двумерном пространстве.  

\includegraphics[scale=0.7]{img/04.png}\\

Базис двумерного ДКП получается поэлементным перемножением базисных функций одномерных случаев.

Затем, после выполнения ДКП выполняется квантование. Квантование это этап, на котором происходит основная потеря информации за счет округления несущественных, высокочастотных ДКП-коэффициентов. Квантование зависит от двух параметров: номера класса и числа квантования. Номер класса влияет на квантование на очень грубом уровне, выбор уровня класса обычно основывается на значение коэффициента AC. Число квантования влияет на масштаб квантования на более мелком уровне и может быть задано только для целого блока. Всего возможных номеров классов четыре, а чисел квантования девять. Кроме того, на масштаб влияет значение разности коэффициентов AC и DC. Чем она меньше, тем мельче масштаб приходится выбирать. Высокие частоты квантуются грубо, уменьшая детализацию, что приводит к уменьшению размеров кадров. Квантование проводится таким образом, чтобы после сжатия сегмент не превышал объем 2560 бит.

В результате квантования получается 64 элемента, которые кодируются при помощи кодов переменной длины.

\pagebreak

