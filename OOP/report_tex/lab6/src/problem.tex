\CWHeader{Лабораторная работа \textnumero 6}

\CWProblem{
Используя структуры данных, разработанные для предыдущей лабораторной работы (ЛР №5) спроектировать и разработать аллокатор памяти для динамической структуры данных. \par

Цель построения аллокатора -- минимизация вызова операции malloc. Аллокатор должен выделять большие блоки памяти для хранения фигур и при создании новых фигур-объектов выделять место под объекты в этой памяти. \\
Аллокатор должен хранить списки использованных/свободных блоков. Для хранения списка свободных блоков нужно применять динамическую структуру данных (контейнер 2-ого уровня, согласно варианту задания). \\
Для вызова аллокатора должны быть переопределены операторы new и delete у классов-фигур. \\


{\bfseries Фигуры:} трапеция, ромб, пятиугольник. \\
{\bfseries Контейнер 1-ого уровня:} связный список. \\
{\bfseries Контейнер 2-ого уровня:} стек.

}
\pagebreak