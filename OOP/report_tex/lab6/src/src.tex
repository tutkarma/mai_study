\section{Описание}

Аллокатор памяти -- часть программы (как прикладной, так и операционной системы), обрабатывающая запросы на выделение и освобождение оперативной памяти или запросы на включение заданной области памяти в адресное пространство процессора. \\
Основное назначение аллокатора памяти в первом смысле -- реализация динамической памяти. В языке С динамическое выделение памяти производится через функцию malloc. \\
Программисты должны учитывать последствия динамического выделения памяти и дважды обдумать использование функции malloc или оператора new. Легко убедить себя, что вы не делаете так уж много аллокаций, а значит большого значения это не имеет, но такой тип мышления распространяется лавиной по всей команде и приводит к медленной смерти. Фрагментация и потери в производительности, связанные с использование динамической памяти, не будучи пресеченными в зародыше, могут иметь катастрофические трудноразрешаемые последствия в вашем дальнейшем цикле разработки. Проекты, где управление и распределение памяти не продумано надлежащим образом, часто страдают от случайных сбоев после длительной сессии из-за нехватки памяти и стоят сотни часов работы программистов, пытающихся освободить память и реорганизовать ее выделение. \\


\section{Исходный код}

Описание классов фигур и класса-контейнера остается неизменным.\\

\begin{longtable}{|p{7.5cm}|p{7.5cm}|}
\rowcolor{lightgray}
\multicolumn{2}{|c|} {TAllocationBlock.cpp}\\
\hline
TAllocationBlock(int32t size, int32t count);&Конструктор класса\\
\hline
void *Allocate();&Выделение памяти\\
\hline
void Deallocate(void *ptr);&Освобождение памяти\\
\hline
bool Empty();&Проверка, пуст ли аллокатор\\
\hline
int32t Size();&Получение количества выделенных блоков\\
\hline
virtual \textasciitilde TAllocationBlock();&Деконструктор класса\\
\hline
\end{longtable}

\begin{lstlisting}[language=C]

class TAllocationBlock
{
public:
    TAllocationBlock(int32_t size, int32_t count);
    void *Allocate();
    void Deallocate(void *ptr);
    bool Empty();
    int32_t Size();

    virtual ~TAllocationBlock();

private:
    Byte *_used_blocks;
    TStack<void *>_free_blocks;
};

\end{lstlisting}


\section{Консоль}
\begin{alltt}
karma@karma:~/mai_study/OOP/lab6$ ./run
Choose an operation:
1) Add trapeze
2) Add rhomb
3) Add pentagon
4) Delete figure from list
5) Print list
6) Print list with iterator
0) Exit
1
Enter bigger base: 10
Enter smaller base: 10
Enter left side: 10
Enter right side: 10
Enter index = 0
Choose an operation:
1) Add trapeze
2) Add rhomb
3) Add pentagon
4) Delete figure from list
5) Print list
6) Print list with iterator
0) Exit
3
Enter side: 10
Enter index = 0
Choose an operation:
1) Add trapeze
2) Add rhomb
3) Add pentagon
4) Delete figure from list
5) Print list
6) Print list with iterator
0) Exit
2
Enter side: 10
Enter smaller angle: 10
Enter index = 1
Choose an operation:
1) Add trapeze
2) Add rhomb
3) Add pentagon
4) Delete figure from list
5) Print list
6) Print list with iterator
0) Exit
5
idx: 0   Sides =  10, type: pentagon

idx: 1   Smaller base = 10, bigger base = 10, left side = 10, right side = 10, type: trapeze

idx: 2   Side = 10, smaller_angle = 10, type: rhomb


Choose an operation:
1) Add trapeze
2) Add rhomb
3) Add pentagon
4) Delete figure from list
5) Print list
6) Print list with iterator
0) Exit
4
Enter index = 0
Choose an operation:
1) Add trapeze
2) Add rhomb
3) Add pentagon
4) Delete figure from list
5) Print list
6) Print list with iterator
0) Exit
4
Enter index = 0
Choose an operation:
1) Add trapeze
2) Add rhomb
3) Add pentagon
4) Delete figure from list
5) Print list
6) Print list with iterator
0) Exit
6
Side = 10, smaller_angle = 10, type: rhomb
Choose an operation:
1) Add trapeze
2) Add rhomb
3) Add pentagon
4) Delete figure from list
5) Print list
6) Print list with iterator
0) Exit
1
Enter bigger base: 10
Enter smaller base: 10
Enter left side: 10
Enter right side: 10
Enter index = 0
Choose an operation:
1) Add trapeze
2) Add rhomb
3) Add pentagon
4) Delete figure from list
5) Print list
6) Print list with iterator
0) Exit
3
Enter side: 10
Enter index = 1
Choose an operation:
1) Add trapeze
2) Add rhomb
3) Add pentagon
4) Delete figure from list
5) Print list
6) Print list with iterator
0) Exit
5
idx: 0   Smaller base = 10, bigger base = 10, left side = 10, right side = 10, type: trapeze

idx: 1   Side = 10, smaller_angle = 10, type: rhomb

idx: 2   Sides =  10, type: pentagon


Choose an operation:
1) Add trapeze
2) Add rhomb
3) Add pentagon
4) Delete figure from list
5) Print list
6) Print list with iterator
0) Exit
0


\end{alltt}

