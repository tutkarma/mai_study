\section{Описание}

Динамические структуры данных используются в тех случаях, когда мы заранее не знаем, сколько памяти необходимо выделить для нашей программы -- это выясняется только в процессе работы. В общем случае эта структура представляет собо отдельные элементы, связанные между собой с помощью ссылок. Каждый элемент состоит из двух областей памяти: поля данных и ссылок. Ссылки -- это адреса других узлов того же типа, с которыми данный элемент логически связан. При добавлении нового элемента в такую структуру выделяется новый блок памяти и устанавливаются связи этого элемента с уже существующими. \\

Структура данных список является простейшим типом данных динамической структуры, состоящей из узлов. Каждый узел включает в себя в классическом варианте два поля: данные и указатель на следующий узел в списке. Элементы связного списка можно вставлять и удалять произвольным образом. Доступ к списку осуществляется через указатель, который содержит ядрес первого элемента списка, называемого головой списка. \\

Параметры в функцию могут передаваться одним из следующих способов: по значению и по ссылке. При передаче аргументов по значению компилятор создает временную копию объекта, который должен быть передан, и размещает его в области стековой памяти, предназначенной для хранения локальных объектов. Вызываемая функция оперирует именно с этой копией, не оказывая влияния на оригинал объекта. Прототипы функций, принимающих аргументы по значению, предусматривают в качестве параметров указание типа объекта,а не его адреса. Если же необходимо, чтобы функция модифицировала оригинал объекта, используется передача параметров по ссылке. При этом в функцию передается не сам объект, а только его адрес. Таким образом, все модификации в теле функции переданных ей по ссылке аргументов воздействуют на объект. Использование передачи адреса объекта весьма эффективный способ работы с большим числом данных. Кроме того, так как передается адрес, а не сам объект, существенно экономится стековая память. \\


\section{Исходный код}

\begin{longtable}{|p{7.5cm}|p{7.5cm}|}
\hline
\rowcolor{lightgray}
\multicolumn{2}{|c|} {trapeze.cpp}\\
\hline
Trapeze();&Конструктор класса\\
\hline
Trapeze(std::istream \&is);&Конструктор класса из стандартного потока\\
\hline
Trapeze(const Trapeze\& orig);&Конструктор копии класса\\
\hline
double Square();&Площадь фигуры\\
\hline
void Print();&Печать фигуры\\
\hline
\textasciitilde Trapeze();&Деконструктор класса\\
\hline
bool operator ==(const Trapeze \&obj) const;&Переопределенный оператор сравнения\\
\hline
Trapeze\& operator =(const Trapeze \&obj);&Переопределенный оператор копирования\\
\hline
friend std::ostream\& operator <<(std::ostream \&os, const Trapeze \&obj);&Переопределенный оператор вывода в поток std::ostream\\
\hline
friend std::istream\& operator >>(std::istream \&is, Trapeze \&obj);&Переопределенный оператор ввода из потока std::istream\\
\hline
\rowcolor{lightgray}
\multicolumn{2}{|c|} {TListItem.cpp}\\
\hline
TListItem(const Trapeze \&obj);&Конструктор класса\\
\hline
Trapeze GetFigure() const;&Получение фигуры из узла\\
\hline
TListItem* GetNext();&Получение ссылки на следующий узел\\
\hline
TListItem* GetPrev();&Получение ссылки на предыдущий узел\\
\hline
void SetNext(TListItem *item);&Установка ссылки на следующий узел\\
\hline
void SetPrev(TListItem *item);&Установка ссылки на предыдущий узел\\
\hline
friend std::ostream\& operator<<(std::ostream \&os, const TListItem \&obj);&Переопределенный оператор вывода в поток std::ostream\\
\hline
virtual \textasciitilde TListItem(){};&Деконструктор класса\\
\hline
\rowcolor{lightgray}
\multicolumn{2}{|c|} {TList.cpp}\\
\hline
TList();&Конструктор класса\\
\hline
void Push(Trapeze \&obj);&Добавление фигуры в список\\
\hline
Trapeze Pop();&Получение фигуры из списка\\
\hline
const bool IsEmpty() const;&Проверка, пуст ли список\\
\hline
uint32t GetLength();&Получение длины списка\\
\hline
friend std::ostream\& operator<<(std::ostream \&os, const TList \&list);&Переопределенный оператор вывода в поток std::ostream\\
\hline
virtual \textasciitilde TList();&Деконструктор класса\\
\hline
\end{longtable}

\begin{lstlisting}[language=C]

class TList
{
public:
    TList();
    void Push(Trapeze &obj);
    const bool IsEmpty() const;
    uint32_t GetLength();
    Trapeze Pop();
    friend std::ostream& operator<<(std::ostream &os, const TList &list);
    virtual ~TList();

private:
    uint32_t length;
    TListItem *head;

    void PushFirst(Trapeze &obj);
    void PushLast(Trapeze &obj);
    void PushAtIndex(Trapeze &obj, int32_t ind);
    Trapeze PopFirst();
    Trapeze PopLast();
    Trapeze PopAtIndex(int32_t ind);
};

class TListItem
{
public:
    TListItem(const Trapeze &obj);

    Trapeze GetFigure() const;
    TListItem* GetNext();
    TListItem* GetPrev();
    void SetNext(TListItem *item);
    void SetPrev(TListItem *item);
    friend std::ostream& operator<<(std::ostream &os, const TListItem &obj);

    virtual ~TListItem(){};

private:
    Trapeze item;
    TListItem *next;
    TListItem *prev;
};

class Trapeze : public Figure
{
public:
    Trapeze();
    Trapeze(std::istream &is);
    Trapeze(int32_t small_base, int32_t big_base, int32_t l_side, int32_t r_side);
    Trapeze(const Trapeze &orig);

    bool operator ==(const Trapeze &obj) const;
    Trapeze& operator =(const Trapeze &obj);
    friend std::ostream& operator <<(std::ostream &os, const Trapeze &obj);
    friend std::istream& operator >>(std::istream &is, Trapeze &obj);

    double Square() override;
    void Print() override;
    virtual ~Trapeze();

private:
    int32_t small_base;
    int32_t big_base;
    int32_t l_side;
    int32_t r_side;
};



\end{lstlisting}


\section{Консоль}
\begin{alltt}
karma@karma:~/mai_study/OOP/lab2$ valgrind --leak-check=full ./run
==4059== Memcheck, a memory error detector
==4059== Copyright (C) 2002-2015, and GNU GPL'd, by Julian Seward et al.
==4059== Using Valgrind-3.11.0 and LibVEX; rerun with -h for copyright info
==4059== Command: ./run
==4059== 
Choose an operation:
1) Add trapeze
2) Delete trapeze from list
3) Print list
0) Exit
3
The list is empty.

Choose an operation:
1) Add trapeze
2) Delete trapeze from list
3) Print list
0) Exit
1
Enter bigger base: 5
Enter smaller base: 2
Enter left side: 1
Enter right side: 1
Enter index = 0
Choose an operation:
1) Add trapeze
2) Delete trapeze from list
3) Print list
0) Exit
1
Enter bigger base: 
4
Enter smaller base: 3
Enter left side: 3
Enter right side: 3
Enter index = 0
Choose an operation:
1) Add trapeze
2) Delete trapeze from list
3) Print list
0) Exit
1
Enter bigger base: 5
Enter smaller base: 5
Enter left side: 5
Enter right side: 5
Enter index = 1
Choose an operation:
1) Add trapeze
2) Delete trapeze from list
3) Print list
0) Exit
1
Enter bigger base: 2
Enter smaller base: 2
Enter left side: 2
Enter right side: 2
Enter index = 2
Choose an operation:
1) Add trapeze
2) Delete trapeze from list
3) Print list
0) Exit
3
idx: 0   (3 4 3 3)

idx: 1   (2 5 1 1)

idx: 2   (5 5 5 5)

idx: 3   (2 2 2 2)


Choose an operation:
1) Add trapeze
2) Delete trapeze from list
3) Print list
0) Exit
2
Enter index = 1
Choose an operation:
1) Add trapeze
2) Delete trapeze from list
3) Print list
0) Exit
2
Enter index = 1
Choose an operation:
1) Add trapeze
2) Delete trapeze from list
3) Print list
0) Exit
2
Enter index = 1
Choose an operation:
1) Add trapeze
2) Delete trapeze from list
3) Print list
0) Exit
3
idx: 0   (3 4 3 3)


Choose an operation:
1) Add trapeze
2) Delete trapeze from list
3) Print list
0) Exit
1
Enter bigger base: 3
Enter smaller base: 3
Enter left side: 3
Enter right side: 3
Enter index = 0
Choose an operation:
1) Add trapeze
2) Delete trapeze from list
3) Print list
0) Exit
1
Enter bigger base: 2
Enter smaller base: 2
Enter left side: 2
Enter right side: 2
Enter index = 1
Choose an operation:
1) Add trapeze
2) Delete trapeze from list
3) Print list
0) Exit
3
idx: 0   (3 3 3 3)

idx: 1   (3 4 3 3)

idx: 2   (2 2 2 2)


Choose an operation:
1) Add trapeze
2) Delete trapeze from list
3) Print list
0) Exit
0
==4059== 
==4059== HEAP SUMMARY:
==4059==     in use at exit: 72,704 bytes in 1 blocks
==4059==   total heap usage: 9 allocs, 8 frees, 75,040 bytes allocated
==4059== 
==4059== LEAK SUMMARY:
==4059==    definitely lost: 0 bytes in 0 blocks
==4059==    indirectly lost: 0 bytes in 0 blocks
==4059==      possibly lost: 0 bytes in 0 blocks
==4059==    still reachable: 72,704 bytes in 1 blocks
==4059==         suppressed: 0 bytes in 0 blocks
==4059== Reachable blocks (those to which a pointer was found) are not shown.
==4059== To see them, rerun with: --leak-check=full --show-leak-kinds=all
==4059== 
==4059== For counts of detected and suppressed errors, rerun with: -v
==4059== ERROR SUMMARY: 0 errors from 0 contexts (suppressed: 0 from 0)


\end{alltt}

