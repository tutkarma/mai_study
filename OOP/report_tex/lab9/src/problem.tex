\CWHeader{Лабораторная работа \textnumero 9}

\CWProblem{
Используя структуры данных, разработанные для лабораторной работы №6 (контейнер 1-ого уровня и классы-фигуры) необходимо разработать:

\begin{itemize}
\item{Контейнер второго уровня с использованием шаблонов.}
\item{Реализовать с помощью лямбда-выражений набор команд, совершающих операции над контейнером 1-ого уровня: генерация фигур со случайными значениями параметров, печать контейнера на экран, удаление элементов со значением площади меньше определенного числа.}
\item{В контейнер второго уровня поместить цепочку команд.}
\item{Реализовать цикл, который проходит по всем командам в контейнере второго уровня и выполняет их, применяя к конейнеру первого уровня.}
\end{itemize}
\\

Для создания потоков ипользовать механизмы:
\begin{itemize}
\item{future}
\item{packaged task/async}
\end{itemize}
\\
Для обеспечения потокобезопасности структур использовать механизмы:
\begin{itemize}
\item{mutex}
\item{lock quard}
\end{itemize}


{\bfseries Фигуры:} трапеция, ромб, пятиугольник. \\
{\bfseries Контейнер 1-ого уровня:} связный список. \\
{\bfseries Контейнер 2-ого уровня:} стек. \\


}
\pagebreak